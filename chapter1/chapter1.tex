\setchapterpreamble[u]{\margintoc}
\chapter{Introduction}

\red{very general introduction}

This thesis is devided in four parts. First we give the necessary background on the spin properties of \Er and \W in a \Ca host crystal. We also describe the interaction between the \Er and a superconducting resonator coupled inductively and the unique detection techniques used in this work, based on microwave photon counting. The second and third parts present the experimental results on the detection of \W nuclear spins and how to use them as quantum register. The last part we discuss the high-resolution NMR spectroscopy of a single \Nb nuclear spin with $I=9/2$.

\section{Of electron and nuclear spins}

Erbium belongs to the family of atoms known as lanthanides. These atoms have gathered significant interest as potential enablers for key quantum technologies such as quantum repeaters, memories and transducers. The unique properties of these atoms can be traced to their electronic structure. Their incomplete $4f$ valence shell is shielded by the outermost $5s$ and $5p$ closed shells from interacting with the environment, ligand electrons and lattice vibrations. This is particularly useful since these atoms can be placed in a crystalline host, where micro- and nanostructures can be fabricated to enhance their interaction with light, while maintaining long optical and spin coherences. In addition, Erbium has an optical transition at 1.5~\textmu m, in the low-loss band of optical fiber, which can be exploited for integration in current network infrastructure.

Half of the lathanides, including erbium, are Krammers ions, and have an effective spin-1/2 ground state when inside of a crystalline host. In this work, we use \Ca with a very low concentration of \Er of 3.1 ppb. \refsec{erbium_ions_in_ca} is dedicated to the understanding of the energy level structure of \Er:\Ca and the limitations of the spin-1/2 approximation.

The nuclear spin environment of \Ca is dominated by \W, which has a nuclear spin-1/2 with a very low gyromagnetic factor and natural abundance of 14\%

\section{Detection of individual nuclear spins and dynamical nuclear polarization}

In \red{refsec} we describe the experimental results regarding the detection and characterization of single-nuclear spins in \Ca. Standard NMR techniques require the combined signal of \red{look} of spins to achieve realistic SNR for detection. Detection of single-nuclear spins faces numerous challenges which can be traced to the low gyromagnetic factor of nuclei. 
Using mechanical and atomic spin sensors, the sensitivity has been pushed to directly detect small ensembles of a few hundreds. 
 

\section{A register of two nuclear spin qubits}


\section{High-resolution spectroscopy on a 9/2 nuclear spin}


