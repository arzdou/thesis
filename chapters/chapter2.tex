\chapter{Background}

\section{Erbium ions in Solids}

% Introduction to rare-earths, brief description of the ions, their properties and their applications
Erbium belongs to the family of atoms known as lanthanides. These atoms have gathered significant interest as potential enablers for key quantum technologies such as quantum repeaters, memories and transducers. The unique properties of these atoms can be traced to their electronic structure. Their incomplete $4f$ valence shell is shielded by the outermost $5s$ and $5p$ closed shells from interacting with the environment, ligand electrons and lattice vibrations. This results in long optical and spin coherences which can be exploited in quantum devices. In addition, Erbium has an optical transition at 1.5~\textmu m, at the low loss band of optical fiber, which can be exploited for integration in current network infrastructure.

When a lanthanide enters a crystal, it does so with a positive trivalent ionization state. will be subject to the electrostatic field generated by the electronic clouds of the neighboring atoms. The crystal-field breaks the spherical symmetry, which lifts the degeneracy of the free-ion states. However, the $4f$ electrons of a lanthanide are highly localized and shielded; the crystal-field interaction can be treated as a perturbation to the free-ion energy levels. This is particullarly useful since microstructures can be fabricated on the crystal to enhance the coupling to light. 

% Description of the spin Hamiltonian. Copy and paste from the paper and extend the discussion

When introduced in a crystal, the energy structure of an \Er impurity can be well modeled by a perturbation to the free-ion Hamiltonian by the electrostatic interaction of the crystal-field. The ground state of the free-ion Hamiltonian is a 16-fold degenerate level and the first excited state is separated by an optical transition at 1.5~\textmu m. Consequently, at cryogenic temperatures, only the ground state will be populated, which can be effectively approximated as a spin $J=15/2$.

%The host crystal used in this work is \Ca. In this crystal, \Er enters with a 3+ charge state in a Ca$^{2+}$ site. The substituting site belongs to the $S_4$ point group, which notably lacks an inversion symmetry. This lack of inversion has deep consequences on the interaction of the ion with electric fields that will be explored later. 

Under a magnetic field $\mathbf{B_0}$, the Hamiltonian of the ground $J=15/2$-multiplet can be written as 

\begin{equation}
    \mathcal{H}_J = \mu_B \ g_J\mathbf{B_0}\cdot \mathbf{J} + \mathcal{H}_{\mathrm{cf}}
\end{equation}

The first term corresponds to the Zeeman interaction, where $\mu_B\approx2\pi\cdot14$~GHz/T corresponds to the Bohr magneton and $g_J=6/5$ is the Landé factor of the \Er ground state. The second term is the crystal-field Hamiltonian, which is generally described in terms of the extended Stevens operators $\hat{O}^q_k$ with $k=2,4,6,\dots$ and $q \in \{-k,\dots, k\}$ \cite{abragam_electron_2012, stevens_matrix_1952}

\begin{equation}
    \mathcal{H}_{\mathrm{cf}} = \sum_{k}^{2,4,6,\dots}\sum_{q}^{-k,...k} B_k^q \hat{O}^q_k \ ,
\end{equation}

Where $B_k^q$ are the crystal field parameters. For \Er:\Ca these values can be found in \cite{enrique_optical_1971} along with the renormalization factors \cite{erath_crystal_1961}. The effect of the crystal-field is the mixing of states with different $J_z$. Kramers' theorem \cite{kramers_general_1930} states that if the number of electrons in the $4f$ shell is odd, then all crystal field levels are at least doublets, which is the case for \Er with $N=11$. This, the crystal-field breaks the 16-fold degenerate state into eight doublets ($Z_{1,\dots,8}$). 

\section{Spins coupled to a cavity}
Spins coupled to a cavity text here.